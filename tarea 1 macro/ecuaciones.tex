\documentclass[12pt]{article}
\usepackage[utf8]{inputenc}
\usepackage{amsmath}
\usepackage[spanish,es-tabla]{babel}
\usepackage{graphicx}
\usepackage{amssymb, amsmath, amsbsy} 
\textwidth 16cm
\hoffset -2cm
\providecommand{\abs}[1]{\lvert#1\rvert}


\title{\textbf{Tarea 1} \\Macroeconomía II}
\author{Pablo Herrera -Tomás Mendoza - Pablo Ramírez}
\date{Mayo 2021}
\begin{document}
	
	\maketitle
	

	
		

	
	Dado que  $U(c_t, l_t)=\gamma \ln c_t+\eta(1-e_tl_t)-\dfrac{\theta}{2}(e_t-1)^2 $ entonces resolvemos el siguiente problema:
		\begin{equation}
		\begin{array}{rrclcl}
			\displaystyle \max_{\{c_t,l_t\}_{t=0}^\infty} & \multicolumn{3}{l}{\gamma \ln c_t+\eta(1-e_tl_t)-\dfrac{\theta}{2}(e_t-1)^2}\nonumber\\
			
		\end{array}
	\end{equation} 
	
	\begin{equation}
			\begin{array}{rrclcl}
			\displaystyle st: & \multicolumn{3}{l}{c_{t}+i_{t}=y_{t}}\nonumber\\
					\end{array}
	\end{equation}
			\begin{equation}
	y_t=A_t k_t^\alpha (e_tl_t)^{1-\alpha} \nonumber
		\end{equation}
	\begin{equation}
	k_{t+1}=(1-\delta)k_{t}+\xi_t i_{t}\nonumber
	\end{equation}
		\begin{equation}
			\gamma, \eta, \theta >0\nonumber
		\end{equation} \\
	
	
	
 \newpage Las ecuaciones resultantes del proceso de optimización son las siguientes:	
 


	\begin{equation}
		E_{t-1}[\dfrac{\gamma}{c_t}(1-\alpha)\dfrac{y_t}{e_t l_t}-\eta]=0
	\end{equation}\\

	\begin{equation}
\dfrac{\gamma}{c_t}(1-\alpha)\dfrac{y_t}{e_t l_t}-\eta=\dfrac{\theta(e_t-1)}{l_t}
		\end{equation} \\
	
	\begin{equation}
	\dfrac{1}{c_t\xi_t}=\beta[\alpha\dfrac{\gamma}{c_{t+1}}\dfrac{y_{t+1}}{k_{t+1}}+(1-\delta) \dfrac{\gamma}{c_{t+1}} \dfrac{1}{xi_{t+1}}
	\end{equation}\\

\begin{equation}
	y_t=c_t+i_t
\end{equation}\\

\begin{equation}
	k_{t+1}=(1-\delta)k_t+\xi_ti_t
\end{equation}\\

\begin{equation}
	y_t=A_t k_t^\alpha (e_tl_t)^{1-\alpha}
\end{equation}\\

\begin{equation}
	\ln A_t=\rho \ln A_{t-1}+u_1
\end{equation}\\

\begin{equation}
		\ln \xi_t=\rho_\xi \ln \xi_{t-1}+u_2
\end{equation}\\

\begin{equation}
	u_1 \sim N(o,\sigma_A^2) \nonumber
\end{equation}\\

\begin{equation}
	u_2 \sim N(o,\sigma_\xi^2) \nonumber
\end{equation}\\
\end{document}